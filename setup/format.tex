% !Mode:: "TeX:UTF-8"
\usepackage[utf8]{inputenc}
%%是否打印答案的布尔变量
\newboolean{printanswers}
\newboolean{twopagenum}
%%定义试卷题头必要的宏
\makeatletter
\def\coursecode#1{\def\@coursecode{#1}}\def\@coursecode{}%课程代码
\def\coursename#1{\def\@coursename{#1}}\def\@coursename{}%课程名称
\def\courseyear#1{\def\@courseyear{#1}}\def\@courseyear{}%课程年份
\def\courseterm#1{\def\@courseterm{#1}}\def\@courseterm{}%考试学期
\def\specialty#1{\def\@specialty{#1}}\def\@specialty{}%专业
\def\grade#1{\def\@grade{#1}}\def\@grade{}%年级
\def\exammode#1{\def\@exammode{#1}}\def\@exammode{}%考核方式
\def\examtime#1{\def\@examtime{#1}}\def\@examtime{}%考试时量
\def\examtype#1{\def\@examtype{#1}}\def\@examtype{}%考核类型
\makeatother
%%中英文字体设置
\setCJKfamilyfont{zhsong}[AutoFakeBold = {2.17}]{SimSun}% 宋体字
\renewcommand*{\songti}{\CJKfamily{zhsong}}%加了这条命令才能宋体加粗显示
\setlength{\headheight}{20pt}%页眉高度
\setlength{\baselineskip}{24pt}%设置行间距
\setmainfont{Times New Roman}%西文主要字体设置为Times New Roman
\setsansfont{Arial}
%%大题计数器
\newcounter{dati}%设计一个自己的计数器,名字叫dati
\setcounter{dati}{0}%计数器初始赋值为零
%%设置有无分隔线
%\columnseprule=0.4pt%试卷分隔线的宽度
%%定义试卷题头
\makeatletter
\newcommand{\examtitle}[7]{
	\ifthenelse{\boolean{printanswers}}
	{
		\heiti\zihao{-3}%设置字体字号小三黑体
		湖南师范大学{\@courseyear}~学年第{\@courseterm}学期~{\@specialty}~{\@grade}~级\heiti\zihao{5}{期末/补考/重修}\heiti\zihao{-3}课程\\
		\uline{\@coursename}考核试题标准答案及评分细则
		\newline
		\vspace{2mm}
		\zihao{-4}\songti
		\bf{课程代码:~{\@coursecode}~~考核方式:~{\@exammode}~~~~~~~考试时量:~{\@examtime}~分钟~~试卷类型:~{\@examtype}}
	}
	{\heiti\zihao{-3}%设置字体字号小三黑体
		湖南师范大学工程与设计学院\\
		{\@courseyear}~学年第{\@courseterm}学期~{\@specialty}~{\@grade}~级\\
		《{\@coursename}》课程期末/补考考试试题
		\newline
		\vspace{2mm}
		\zihao{-4}\songti
		课程代码:~{\@coursecode}~~考核方式:~{\@exammode}~~~~~~~考试时量:~{\@examtime}~分钟~~试卷类型:~{\@examtype}
		\examscoretable{#1}{#2}{#3}{#4}{#5}{#6}{#7}
	}
}
\makeatother
%%定义总计分表头
\newcommand{\examscoretable}[7]{
	\begin{table}[h]
		\zihao{-4}\songti%设置字体为小四宋体;
		\centering%居中
		\setlength{\tabcolsep}{2.8mm}
		\renewcommand\arraystretch{1.5}%表格缩放1.5倍;
		\begin{tabular}{*{11}{|c}|}%总共11个格子;
			\hline
			\raisebox{-3pt}{题号} & \raisebox{-3pt}{一} &\raisebox{-3pt}{二}&\raisebox{-3pt}{三} &\raisebox{-3pt}{ 四 }& \raisebox{-3pt}{五} & \raisebox{-3pt}{六} & \raisebox{-3pt}{七} & \raisebox{-3pt}{总分}&\raisebox{-3pt}{合分人}&\raisebox{-3pt}{复查人} \\[2mm]
			\hline
			\raisebox{-3pt}{应得分}&\raisebox{-3pt}{#1}&\raisebox{-3pt}{#2}&\raisebox{-3pt}{#3}&\raisebox{-3pt}{#4}&\raisebox{-3pt}{#5}&\raisebox{-3pt}{#6}&\raisebox{-3pt}{#7}&\raisebox{-3pt}{100}&\multirow{2}{*}{}& \\[2mm]% 各个小题分值
			\cline{1-9}
			\raisebox{-3pt}{实得分}&&&&&&&&&& \\[2mm]
			\hline
		\end{tabular}
	\end{table}
}
%%定义常用命题计分表头
\newcommand{\makepart}[3]{%总共控制3个参数
	\stepcounter{dati}%每进来一次,大题计数器加一;
	\ifthenelse{\boolean{printanswers}}
	{
		\noindent\textbf{\zihao{-4}\songti\zhnum{dati}、#1(每小题~#2~分,共~#3~分)}}
	{
		\noindent\hspace{-0.4cm}\begin{minipage}{16cm}
			\parpic[l]{
				\noindent
				\setlength{\tabcolsep}{2.8mm}
				\renewcommand\arraystretch{1.1}
				\begin{tabular}{|c|c|c|}
					\hline \raisebox{-3pt}{得分}&\raisebox{-3pt}{评卷人}&\raisebox{-3pt}{复查人} \\[2mm]
					\hline & & \\[2mm]
					\hline
				\end{tabular}
			}
			\noindent\textbf{\zihao{-4}\songti\zhnum{dati}、#1(每小题~#2~分,共~#3~分)}
		\end{minipage}\\\\
	}

}
%%定义填空题命题计分表头
\newcommand{\makeparttk}[3]{%总共控制3个参数
	\stepcounter{dati}%每进来一次,大题计数器加一;
		\ifthenelse{\boolean{printanswers}}
		{
		\noindent\textbf{\zihao{-4}\songti\zhnum{dati}、#1(每空~#2~分,共~#3~分)}
		}
		{
		\noindent\hspace{-0.4cm}
		\begin{minipage}{16cm}
		\parpic[l]{
			\noindent
			\setlength{\tabcolsep}{2.8mm}
			\renewcommand\arraystretch{1.1}
			\begin{tabular}{|c|c|c|}
				\hline \raisebox{-3pt}{得分}&\raisebox{-3pt}{评卷人}&\raisebox{-3pt}{复查人} \\[2mm]
				\hline & & \\[2mm]
				\hline
			\end{tabular}
		}
		\noindent\textbf{\zihao{-4}\songti\zhnum{dati}、#1(每空~#2~分,共~#3~分)}
		\end{minipage}\\\\
		}
	
}
%%定义单选题命题计分表头
\newcommand{\makepartdx}[3]{%总共控制3个参数
	\stepcounter{dati}%每进来一次,大题计数器加一;
	\ifthenelse{\boolean{printanswers}}
	{
		\noindent\textbf{\zihao{-4}\songti\zhnum{dati}、#1(在本题的每一小题的备选答案中,只有一个答案是正确的,请把你认为正确答案的题号,填入题干的括号内。多选不给分。每题#2分,共#3分)}
	}
	{	\noindent\hspace{-0.4cm}\begin{minipage}{14.5cm}
			\parpic[l]{
				\noindent
				\setlength{\tabcolsep}{2.8mm}
				\renewcommand\arraystretch{1.1}
				\begin{tabular}{|c|c|c|}
					\hline \raisebox{-3pt}{得分}&\raisebox{-3pt}{评卷人}&\raisebox{-3pt}{复查人} \\[2mm]
					\hline & & \\[2mm]
					\hline
				\end{tabular}
			}
			\noindent\textbf{\zihao{-4}\songti\zhnum{dati}、#1(在本题的每一小题的备选答案中,只有一个答案是正确的,请把你认为正确答案的题号,填入题干的括号内。多选不给分。每题#2分,共#3分)}
		\end{minipage}\\\\
	}

}
%%定义不常用命题题头
\newcommand{\makepartnoteq}[3]{%总共控制3个参数
	\stepcounter{dati}%每进来一次,大题计数器加一;
		\ifthenelse{\boolean{printanswers}}
		{
		\noindent\textbf{\zihao{-4}\songti\zhnum{dati}、#1(本大题共~#2~小题,共~#3~分)}
		}
		{\noindent\hspace{-0.4cm}\begin{minipage}{16cm}
				\parpic[l]{
					\noindent
					\setlength{\tabcolsep}{2.8mm}
					\renewcommand\arraystretch{1.1}
					\begin{tabular}{|c|c|c|}
						\hline \raisebox{-3pt}{得分}&\raisebox{-3pt}{评卷人}&\raisebox{-3pt}{复查人} \\[2mm]
						\hline & & \\[2mm]
						\hline
					\end{tabular}
				}
			\noindent\textbf{\zihao{-4}\songti\zhnum{dati}、#1(本大题共~#2~小题,共~#3~分)}
			\end{minipage}\\\\
		}

}
%%不同页码输出格式
\newcommand{\pagenumstyle}{%\ifthenelse{判断条件}{肯定结构}{否定结构}
\ifthenelse{\boolean{twopagenum}}{%一面的两栏都有页码
	\pagestyle{fancy}
	\renewcommand{\headrulewidth}{0pt}
	\fancyfoot[CO,CE]{共~\the\numexpr2*\getpagerefnumber{mylastpage}\relax~页~~~~第~\the\numexpr2*\value{page}-1\relax~页~~\hspace*{.4\linewidth}共~\the\numexpr2*\getpagerefnumber{mylastpage}\relax~页~~~~第~\the\numexpr2*\value{page}\relax~页}
	\ifthenelse{\boolean{printanswers}}
	{
	}
	{
	\fancyhead[RE]{\leavevmode\vbox to0pt{
			\vss\rlap{\putzdxx }\vskip -26cm }}%奇数页眉的右边
	\fancyhead[LO]{\leavevmode\vbox to0pt{
			\vss\rlap{\putzdx }\vskip -26cm }} %偶数页眉的左边
	}
}
{%一面只一个页码
\pagestyle{fancy}
\renewcommand{\headrulewidth}{0pt}
\fancyfoot[CO,CE]{第~\thepage~页~~共~\pageref{LastPage}~页}
\ifthenelse{\boolean{printanswers}}
{
}
{
\fancyhead[RE]{\leavevmode\vbox to0pt{
		\vss\rlap{\putzdxx }\vskip -26cm }} %奇数页眉的右边                       
\fancyhead[LO]{\leavevmode\vbox to0pt{
		\vss\rlap{\putzdx }\vskip -26cm }} %偶数页眉的左边
}
}
}
%%正面姓名栏+密封线
\newsavebox{\zdxa}
\sbox{\zdxa}
{\parbox{27cm}{\centering \zihao{4}\songti \hspace{1cm}
		学院:\underline{\makebox[28mm][c]{}}~~~ 专业/班级:\underline{\makebox[28mm][c]{}}~~~ 姓名:\underline{\makebox[28mm][c]{}}~~~ 学号:\underline{\makebox[28mm][c]{}}~~~座位号:\underline{\makebox[10mm][c]{}} \\
		\vspace{-1mm}
%		请在所附答题纸上空出密封位置。并填写学院、专业/班级、学号、姓名和座位号\\
		\zihao{5}\songti
		\dotfill\\
		\vspace{-1mm}%调整装订虚线位置,上移1mm
		装~~~~~~~~订~~~~~~~~线~~~~~~~~内~~~~~~~~不~~~~~~~~要~~~~~~~~答~~~~~~~~题\\
		\vspace{-3mm}%调整装订虚线位置,上移3mm
\dotfill
}
}
%%背面装订线
\newsavebox{\zdxb}
\sbox{\zdxb}
{\parbox{27cm}{\centering%指定方框高度
		\zihao{5}\songti
		\dotfill\\
		\vspace{-1mm}%调整装订虚线位置,上移1mm
		装~~~~~~~~订~~~~~~~~线~~~~~~~~内~~~~~~~~不~~~~~~~~要~~~~~~~~答~~~~~~~~题\\
		\vspace{-3mm}%调整装订虚线位置,上移3mm
		\dotfill
}
}
%%诚信考试框
\newsavebox{\zdxc}
\sbox{\zdxc}{
	\begin{tabular}{|c|}
		\hline \raisebox{-3pt}{\textbf{\songti\zihao{-3} 诚信参考,考试舞弊将带来严重后果!}}  \\[2mm]
		\hline
	\end{tabular}
}
%%教务办填写框
\newsavebox{\zdxd}
\sbox{\zdxd}{
	\songti\zihao{-4}
	\renewcommand\arraystretch{0.8}\begin{tabular}{|c|}
		\hline \raisebox{-3pt}{教务办填写}\\[2mm]
		\hline \raisebox{-3pt}{\underline{~~~~~~}年\underline{~~~~}月\underline{~~~~}日} \\[2mm]
		\hline \raisebox{-3pt}{考~~~~试~~~~用}\\[2mm]
		\hline
	\end{tabular}%~~~~
}
%设置正面密封线的摆放位置
\newcommand{\putzdx}{%设置正面密封线的摆放位置
		\hspace{-20mm}\parbox{1cm}{
			\rotatebox[origin=c]{90}{
				\usebox{\zdxa}
		}}
}
%设置背面密封线的摆放位置
\newcommand{\putzdxx}{%设置背面密封线的摆放位置
	\hspace{4mm}\parbox{1cm}{
		\rotatebox[origin=c]{-90}{
			\usebox{\zdxb}
	}}
}
%%放置诚信和教务填写表格
\def\examothertable{
	\ifthenelse{\boolean{printanswers}}{}{
	\begin{figure}[t]
	\begin{center}
		\begin{picture}(1,1)(230,0)% 第一个括号定义尺寸,后一个括号定位坐标
		\put(12,0){\rotatebox[origin=c]{0}{\usebox{\zdxc}}}%诚信考试框
		\put(340,-15){\rotatebox[origin=c]{0}{\usebox{\zdxd}}}%教务办填写框
		\end{picture}
	\end{center}
	\end{figure}}
	
}
%%\newcommand\cmd[参数个数][参数的默认值]{命令的定义}
%%填空题用横线
%%根据答案留大小,前后默认多留0.5cm;\tkhx[1]{}答案前后多留1cm;
\newcommand{\tkhx}[2][0.5]{\;\uline{
		\hspace*{#1 cm}
		\ifthenelse{\boolean{printanswers}}{#2}{\hphantom{#2}}
		\hspace*{#1 cm}
}
}
%%填空题用括号
%%根据答案留大小,前后默认多留0.5cm;\tkhx[1]{}答案前后多留1cm;
\newcommand{\tkkh}[2][0.5]{\;(
		\hspace*{#1 cm}
		\ifthenelse{\boolean{printanswers}}{#2}{\hphantom{#2}}
		\hspace*{#1 cm})
}
%%选择题选项命令
%%\xx{}{}{}{} 四选项;
%%选项自动对齐;
\newlength{\la}
\newlength{\lb}
\newlength{\lc}
\newlength{\ld}
\newlength{\lee}
\newlength{\lf}
\newlength{\lhalf}
\newlength{\lquarter}
\newlength{\lmax}
\newcommand{\xx}[4]{\\[.5pt]%  
	\settowidth{\la}{A、#1~~~}
	\settowidth{\lb}{B、#2~~~}
	\settowidth{\lc}{C、#3~~~}
	\settowidth{\ld}{D、#4~~~}
	\ifthenelse{\lengthtest{\la > \lb}}
	{\setlength{\lmax}{\la}}{\setlength{\lmax}{\lb}}
		\ifthenelse{\lengthtest{\lmax < \lc}}  {\setlength{\lmax}{\lc}}{}
		\ifthenelse{\lengthtest{\lmax < \ld}}{\setlength{\lmax}{\ld}}{}
		\setlength{\lhalf}{0.5\linewidth}
		\setlength{\lquarter}{0.25\linewidth}
		\ifthenelse{\lengthtest{\lmax > \lhalf}}
		{\noindent{}A、#1 \\ B、#2 \\ C、#3 \\ D、#4 }  {  \ifthenelse{\lengthtest{\lmax > \lquarter}}  
			{\noindent
			\makebox[\lhalf][l]{A、#1~~~}% 
			\makebox[\lhalf][l]{B、#2~~~}\\%
			\makebox[\lhalf][l]{C、#3~~~}%
			\makebox[\lhalf][l]{D、#4~~~}}% 
			{\noindent\makebox[\lquarter][l]{A、#1~~~}% 
			\makebox[\lquarter][l]{B、#2~~~}%     
			\makebox[\lquarter][l]{C、#3~~~}%      
			\makebox[\lquarter][l]{D、#4~~~}}
		}}
%%\xxiii{}{}{} 三选项;
\newcommand{\xxiii}[3]{\\[.5pt]%  
	\settowidth{\la}{A、#1~~~}  
	\settowidth{\lb}{B、#2~~~}
	\settowidth{\lc}{C、#3~~~}  
	\ifthenelse{\lengthtest{\la > \lb}}
	{\setlength{\lmax}{\la}}{\setlength{\lmax}{\lb}}  
	\ifthenelse{\lengthtest{\lmax < \lc}}  {\setlength{\lmax}{\lc}}  {}  
	\setlength{\lhalf}{0.5\linewidth} 
	\setlength{\lquarter}{0.25\linewidth}
	\ifthenelse{\lengthtest{\lmax > \lhalf}}  
	{\noindent{}A、#1 \\ B、#2 \\ C、#3 }  {  \ifthenelse{\lengthtest{\lmax > \lquarter}}  
		{\noindent
			\makebox[\lhalf][l]{A、#1~~~}% 
			\makebox[\lhalf][l]{B、#2~~~}\\%
			\makebox[\lhalf][l]{C、#3~~~}}% 
		{\noindent
			\makebox[\lquarter][l]{A、#1~~~}% 
			\makebox[\lquarter][l]{B、#2~~~}%     
			\makebox[\lquarter][l]{C、#3~~~}}
}}
%%\xxv{}{}{}{}{} 五选项;
\newcommand{\xxv}[5]{\\[.5pt]%  
	\settowidth{\la}{A、#1~~~}  
	\settowidth{\lb}{B、#2~~~}
	\settowidth{\lc}{C、#3~~~} 
	\settowidth{\ld}{D、#4~~~}  
	\settowidth{\lee}{E、#5~~~}   
	\ifthenelse{\lengthtest{\la > \lb}}
	{\setlength{\lmax}{\la}}{\setlength{\lmax}{\lb}}  
	\ifthenelse{\lengthtest{\lmax < \lc}}  {\setlength{\lmax}{\lc}}  {} 
	\ifthenelse{\lengthtest{\lmax < \ld}}  {\setlength{\lmax}{\ld}}  {} 
	\ifthenelse{\lengthtest{\lmax < \lee}}  {\setlength{\lmax}{\lee}}  {} 
	\setlength{\lhalf}{0.5\linewidth} 
	\setlength{\lquarter}{0.25\linewidth}
	\ifthenelse{\lengthtest{\lmax > \lhalf}}  
	{\noindent{}A、#1 \\ B、#2 \\ C、#3 \\ D、#4 \\ E、#5}  {  \ifthenelse{\lengthtest{\lmax > \lquarter}}  
		{\noindent
			\makebox[\lhalf][l]{A、#1~~~}% 
			\makebox[\lhalf][l]{B、#2~~~}\\%
			\makebox[\lhalf][l]{C、#3~~~}%
		    \makebox[\lhalf][l]{D、#4~~~}\\%
		    \makebox[\lhalf][l]{E、#5~~~}}% 
		{\noindent
			\makebox[\lquarter][l]{A、#1~~~}% 
			\makebox[\lquarter][l]{B、#2~~~}%     
			\makebox[\lquarter][l]{C、#3~~~}%
		    \makebox[\lquarter][l]{D、#4~~~}\\%
	        \makebox[\lquarter][l]{E、#5~~~}}
}}
%%\xxvi{}{}{}{}{}{} 六选项;
\newcommand{\xxvi}[6]{\\[.5pt]
	\settowidth{\la}{A、#1~~~}
	\settowidth{\lb}{B、#2~~~}
	\settowidth{\lc}{C、#3~~~}
	\settowidth{\ld}{D、#4~~~}
	\settowidth{\lee}{E、#5~~~}
	\settowidth{\lf}{F、#6~~~}
	\ifthenelse{\lengthtest{\la > \lb}}
	{\setlength{\lmax}{\la}}{\setlength{\lmax}{\lb}}  
	\ifthenelse{\lengthtest{\lmax < \lc}}  {\setlength{\lmax}{\lc}}  {} 
	\ifthenelse{\lengthtest{\lmax < \ld}}  {\setlength{\lmax}{\ld}}  {} 
	\ifthenelse{\lengthtest{\lmax < \lee}}  {\setlength{\lmax}{\lee}}  {}
	\ifthenelse{\lengthtest{\lmax < \lf}}  {\setlength{\lmax}{\lf}}  {}
	\setlength{\lhalf}{0.5\linewidth} 
	\setlength{\lquarter}{0.25\linewidth}
	\ifthenelse{\lengthtest{\lmax > \lhalf}}  
	{\noindent{}A、#1 \\ B、#2 \\ C、#3 \\ D、#4 \\ E、#5 \\ F、#6}  {  \ifthenelse{\lengthtest{\lmax > \lquarter}}  
		{\noindent
			\makebox[\lhalf][l]{A、#1~~~}% 
			\makebox[\lhalf][l]{B、#2~~~}\\%
			\makebox[\lhalf][l]{C、#3~~~}%
			\makebox[\lhalf][l]{D、#4~~~}\\%
			\makebox[\lhalf][l]{E、#5~~~}%
		    \makebox[\lhalf][l]{F、#6~~~}}% 
		{\noindent
			\makebox[\lquarter][l]{A、#1~~~}%
			\makebox[\lquarter][l]{B、#2~~~}%
			\makebox[\lquarter][l]{C、#3~~~}%
			\makebox[\lquarter][l]{D、#4~~~}\\%
			\makebox[\lquarter][l]{E、#5~~~}%
		    \makebox[\lquarter][l]{F、#6~~~}}%
}}
%%判断题后面加括号
\newcommand{\pd}[2][1]{\nolinebreak\dotfill\mbox{\raisebox{-1.8pt}
		{$\cdots$}(\makebox[#1 cm][c]{
		\ifthenelse{\boolean{printanswers}}
		{\ifthenelse{\equal{#2}{t}}{$\surd$}{$\times$}}
		{\hphantom{#2}}
		})}}
%%计算题
\newcommand{\js}[2][4]{\par
	\parbox[t][#1cm][t]{\linewidth}{
		\ifthenelse{\boolean{printanswers}}{
		\begin{spacing}{1.5}
		解:\setlength\parindent{2em}#2
		\end{spacing}
		}{}
	}
}
%%简答题
\newcommand{\jd}[2][4]{\par
\parbox[t][#1cm][t]{\linewidth}{
	\ifthenelse{\boolean{printanswers}}{
	\begin{spacing}{1.5}
	答:\setlength\parindent{2em}#2
	\end{spacing}
	}{}
}
}
%%\newcommand\cmd[参数个数]{命令的定义}
%%简答给分点
\newcommand{\jdgf}[1]{
\ifthenelse{\boolean{printanswers}}{\nolinebreak\dotfill\mbox{\raisebox{-1.8pt}{$\cdots$}(#1~分)}}{}
}
